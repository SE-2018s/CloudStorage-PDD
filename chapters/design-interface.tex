\chapter{接口设计}
\section{外部接口}

\subsection{HTTP接口}
xyz云盘系统通过HTTP请求的方式实现API调用。

\subsubsection{初始界面接口}
URL:/api/index

请求类型:GET

参数:无参数
 
返回值:返回初始页面

\subsubsection{登录接口}
URL:/api/login

请求类型:POST

参数:

• username:用户名/邮箱

• pass:密码经过密码学处理之后的字符串
 
返回值:bool类型,表示是否成功,同时设置cookie

\subsubsection{注册接口}
URL:/api/register

请求类型:POST

参数:

• username:用户名

• email:邮箱

• 密码:密码经过密码学处理之后得到的字符串

返回值:bool类型,表示是否注册成功.如果成功,则附带一个分享链接

\subsubsection{忘记密码接口}
URL:/api/forget

请求类型:POST

参数:

• username:用户名/邮箱

返回值:返回bool值表示是否成功.如果用户存在,则服务器向该邮箱发送重设密码的链接.

\subsubsection{登陆后接口}
URL:/api/home

请求类型:GET

参数:

• cookie,无需用户手动输入

返回值:显示登陆后的页面。

\subsubsection{上传接口}
URL:/api/upload

请求类型:POST

参数:

• path:文件的路径

• cookie,无需用户手动输入

返回值:bool类型,表示是否成功。并返回显示的页面

\subsubsection{下载接口}
URL:/api/download

请求类型:GET

参数:

• path:文件路径

• link:可选,可以直接下载链接(尤其是游客)

• cookie:无需用户手动输入

返回值:bool类型,表示是否成功。

\subsubsection{重命名接口}
URL:/api/rename

请求类型:POST
 
参数:

• path:旧文件路径

• path:新文件名

• cookie:无需用户手动输入

返回值:bool类型,表示是否成功。

\subsubsection{粘贴接口}
URL:/api/paste

请求类型:POST

参数:

• path:旧文件路径

• path:新文件路径

• cookie:无需用户手动输入

返回值:bool类型,表示是否成功。并返回页面

\subsubsection{新建文件夹接口}
URL:/api/newDir

请求类型:POST

参数:

• path:文件夹路径

• cookie:无需用户手动输入

返回值:bool类型,表示是否成功。并返回页面

\subsubsection{打开文件夹接口}
URL:/api/openDir

请求类型:GET

参数:

• path:文件夹路径

• cookie:无需用户手动输入

返回值:bool类型,表示是否成功。并返回显示的页面
\subsubsection{返回上一级文件夹接口}
URL:/api/openDir

请求类型:GET

参数:

• path:文件夹路径

• cookie:无需用户手动输入

返回值:bool类型,表示是否成功。并返回显示的页面

\subsubsection{排序显示接口}
URL:/api/sort.do

请求类型:GET

参数:

• path:文件夹路径

• attribute:排序根据的属性

• cookie:无需用户手动输入

返回值:bool类型,表示是否成功。并返回显示的页面

\subsubsection{移入移出回收站接口}
URL:/api/garbage.do

请求类型:POST

参数:

• path:文件路径

• op:表示移入还是移出,还是彻底删除

• cookie:无需用户手动输入

返回值:bool类型,表示是否成功.并返回显示的页面

\subsubsection{移入移出收藏夹接口}
URL:/api/collection

请求类型:POST

参数:

• path:文件路径

• op:表示移入还是移出收藏夹

• cookie:无需用户手动输入

返回值:bool类型,表示是否成功.并返回显示的页面
 
\subsubsection{登录接口}
URL:/api/lookupattribute

请求类型:GET

参数:

• path:文件路径

• cookie:无需用户手动输入

返回值:bool类型,表示是否成功

\subsubsection{压缩/解压缩接口}
URL:/api/zip

请求类型:POST

参数: 

• path:文件路径

• op:表示压缩还是解压缩

• cookie:无需用户手动输入

返回值:bool类型,表示是否成功.并返回显示的页面

\subsubsection{举报接口}
URL:/api/report

请求类型:POST

参数: 

• path:文件路径

• cookie:无需用户手动输入

返回值:bool类型,表示是否成功并返回显示的页面

\subsubsection{搜索接口}
URL:/api/search.do

请求类型:POST

参数: 

• path:文件路径

• cookie:无需用户手动输入

返回值:bool类型,表示是否成功.并返回显示的页面

\subsubsection{生成分享链接接口}
URL:/api/share.do

请求类型:POST

参数: 

• path:文件路径

• cookie:无需用户手动输入

返回值:链接的地址.并返回显示的页面

\subsubsection{导入链接接口}
URL:/api/getshare.do

请求类型:POST

参数: 

• link:要导入的链接

• path:文件夹

• cookie:无需用户手动输入

返回值:bool类型,表示是否成功.并返回显示的页面

{
\color{red}

\subsubsection{好友列表}
URL:/api/friend\_list

请求类型:GET

参数:

• cookie:无需用户手动输入

返回值:当前用户的所有好友

\subsubsection{查找好友}
URL:/api/searchuser

请求类型:GET

参数:

• type:key的类型

• key:查找的关键字

• cookie:无需用户手动输入

返回值:精确匹配的用户结果

\subsubsection{添加好友}
URL:/api/add\_friend

请求类型:POST

参数:

• id:好友ID

• cookie:无需用户手动输入

返回值:是否成功发起添加好友

\subsubsection{删除好友}
URL:/api/delete\_friend

请求类型:POST

参数:

• id:好友ID

• cookie:无需用户手动输入

\subsubsection{开通、续费会员}
URL:/api/make\_vip

请求类型:POST

参数:

• time:开通vip的时长

• cookie:无需用户手动输入

}


\section{内部接口}

\subsection{MODULE.CLIENT(SERVER).1: 通信模块}

\subsubsection{connection create\_connection()}
说明:建立连接

\subsection{MODULE.CLIENT(SERVER).2: 登陆模块}

\subsubsection{user\_login(connection,loginfo)}
参数connection:浏览器和服务器间的连接

参数loginfo:用户登录信息

说明:用户登录账号
class loginfo:

String user\_name:用户名

String password:密码
\subsubsection{user\_register(connection, loginfo)}
参数connection:浏览器和服务器间的连接

参数loginfo:用户注册信息

说明:用户注册账号
\subsubsection{user\_find\_password(connection, user\_name, user\_email)}
参数connection:浏览器和服务器间的连接

参数user\_name:用户名

参数user\_email:用户注册时的电子邮箱

说明:忘记密码
\subsection{MODULE.CLIENT.3: 标签页模块}

\subsubsection{add\_tab(connection, tabinfo)}
参数connection:浏览器和服务器间的连接

参数tabinfo:已经打开的标签页信息

说明:创建标签页
\subsection{MODULE.CLIENT.4: 显示模块}

\subsubsection{show(connection)}
参数connection:浏览器和服务器间的连接

说明:显示标签页
\subsection{MODULE.CLIENT.5(SERVER.4):传输模块}
\subsubsection{upload(connection, file, filepath)}
参数connection:浏览器和服务器间的连接

参数file:要传输的文件

参数filepath:云端文件路径

说明:传输文件到服务器
\subsubsection{download(connection, filepath)}

参数connection:浏览器和服务器间的连接

参数filepath:云端文件路径

说明:传输文件到客户端
\subsection{MODULE.SERVER.3:权限管理模块}
\subsubsection{check\_right(filepath,userId}
参数filepath::所要检查的文件路径.

参数userId:用户名

说明:检查用户名对该文件的权限

\subsection{MODULE.SERVER.5: 分享管理模块}
\subsubsection{create\_share(file\_path, fetch\_password, due\_time}
参数file\_path:要分享的文件的\_path

参数fetch\_password:下载该资源需要的密码

参数due\_time:链接失效日期

说明:生成分享链接

\subsubsection{send\_share(shareInfog's)}
class shareInfo:

    int owner\_path

    str link\_url

    str link\_passward

    time due\_time

\subsection{MODULE.SERVER.6: 收藏夹模块}

\subsubsection{add\_bookmark(bookmark\_path, file\_path)}
参数file\_path:将要添加到收藏夹的文件

参数bookmark\_path:收藏夹路径

说明:移入收藏夹
\subsubsection{delete\_bookmark(file\_path)}
参数file\_path:要删除文件的路径
 
说明:移出收藏夹
\subsection{MODULE.SERVER.7: 文件管理模块}

\subsubsection{insert\_file(file\_path, ceph\_obj)}
参数file\_path:将要插入ceph模块的文件路径

参数ceph\_obj:ceph的一个block对象

说明:添加文件

\subsubsection{drop\_file(ceph\_obj, file\_obj)}
参数ceph\_obj:ceph的一个block对象

说明:移入回收站

参数file\_obj:将要删除的该block里的文件
\subsection{MODULE.SERVER.8: 审核模块}
\subsubsection{report\_file(file\_path)}
参数file\_path:用户所举报的文件路径

说明:用户举报文件非法

\subsubsection{check\_legality(file\_obj)}
参数file\_obj:所要检查的文件路径.

说明:检查文件是否非法


\subsubsection{list(filepath)}
参数:filepath:路径

说明:将指定路径的文件列出来


\subsection{MODULE.SERVER.9:回收站模块}

\subsubsection{delete(userID, filepath)}
参数 userID:执行删除操作的用户

参数 filepath:要删除的文件路径

说明:将该文件置为被删除状态,添加到该用户的回收站列表中

\subsubsection{recover(userID, filepath)}
参数 userID:执行恢复操作的用户

参数 filepath:要恢复的文件路径

说明:将该文件从该用户的回收站列表中移除,置为正常状态

\subsubsection{list(userID)}
参数 userID

说明:列出该用户回收站中所有文件

\subsubsection{clean(userID)}
参数 userID:执行清空操作的用户

说明:清空该用户的回收站列表

\subsection{MODULE.SERVER.10:压缩模块}
\subsubsection{compress(filelist, zipname, level, password)}

参数:filelist:要压缩的所有文件、文件夹

参数:zipname:压缩文件的命名(带路径)

参数:level:压缩级别,从最快压缩速度到最高压缩率

参数:password:要添加的压缩密码

说明:对传入的文件列表创建一个压缩包

\subsubsection{depress(zipname, password)}
参数:zipname:压缩文件的名称(带路径)

参数:password:解压密码,可选

说明:对传入的压缩文件就地解压

\subsection{MODULE.SERVER.11:共享文件夹模块}
\subsubsection{open(foldername, userID)}
参数:foldername:共享文件夹名称

参数:userID:用户ID

说明:根据用户权限判断是否能否进入,若能则调用文件管理模块的接口
 
\subsubsection{list(userID)}
参数:userID:用户ID

说明:返回所有该用户可见的共享文件夹列表



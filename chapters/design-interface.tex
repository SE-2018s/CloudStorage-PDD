\chapter{接口设计}
\section{外部接口}

\subsection{HTTP接口}
xyz云盘系统通过HTTP请求的方式实现API调用。

\subsubsection{初始界面接口}
URL:/api/index

请求类型:GET

参数:无参数
 
返回值:返回初始页面

\subsubsection{登录接口}
URL:/api/login

请求类型:POST

参数:

• username:用户名/邮箱

• pass:密码经过密码学处理之后的字符串
 
返回值:bool类型,表示是否成功,同时设置cookie

\subsubsection{注册接口}
URL:/api/register

请求类型:POST

参数:

• username:用户名

• email:邮箱

• 密码:密码经过密码学处理之后得到的字符串

返回值:bool类型,表示是否注册成功.如果成功,则附带一个分享链接

\subsubsection{忘记密码接口}
URL:/api/forget

请求类型:POST

参数:

• username:用户名/邮箱

返回值:返回bool值表示是否成功.如果用户存在,则服务器向该邮箱发送重设密码的链接.

\subsubsection{登陆后接口}
URL:/api/home

请求类型:GET

参数:

• cookie,无需用户手动输入

返回值:显示登陆后的页面。

\subsubsection{上传接口}
URL:/api/upload

请求类型:POST

参数:

• path:文件的路径

• cookie,无需用户手动输入

返回值:bool类型,表示是否成功。并返回显示的页面

\subsubsection{下载接口}
URL:/api/download

请求类型:GET

参数:

• path:文件路径

• link:可选,可以直接下载链接(尤其是游客)

• cookie:无需用户手动输入

返回值:bool类型,表示是否成功。

\subsubsection{重命名接口}
URL:/api/rename

请求类型:POST
 
参数:

• path:旧文件路径

• path:新文件名

• cookie:无需用户手动输入

返回值:bool类型,表示是否成功。

\subsubsection{粘贴接口}
URL:/api/paste

请求类型:POST

参数:

• path:旧文件路径

• path:新文件路径

• cookie:无需用户手动输入

返回值:bool类型,表示是否成功。并返回页面

\subsubsection{新建文件夹接口}
URL:/api/newDir

请求类型:POST

参数:

• path:文件夹路径

• cookie:无需用户手动输入

返回值:bool类型,表示是否成功。并返回页面

\subsubsection{打开文件夹接口}
URL:/api/openDir

请求类型:GET

参数:

• path:文件夹路径

• cookie:无需用户手动输入

返回值:bool类型,表示是否成功。并返回显示的页面
\subsubsection{返回上一级文件夹接口}
URL:/api/openDir

请求类型:GET

参数:

• path:文件夹路径

• cookie:无需用户手动输入

返回值:bool类型,表示是否成功。并返回显示的页面

\subsubsection{排序显示接口}
URL:/api/sort.do

请求类型:GET

参数:

• path:文件夹路径

• attribute:排序根据的属性

• cookie:无需用户手动输入

返回值:bool类型,表示是否成功。并返回显示的页面

\subsubsection{移入移出回收站接口}
URL:/api/garbage.do

请求类型:POST

参数:

• path:文件路径

• op:表示移入还是移出,还是彻底删除

• cookie:无需用户手动输入

返回值:bool类型,表示是否成功.并返回显示的页面

\subsubsection{移入移出收藏夹接口}
URL:/api/collection

请求类型:POST

参数:

• path:文件路径

• op:表示移入还是移出收藏夹

• cookie:无需用户手动输入

返回值:bool类型,表示是否成功.并返回显示的页面
 
\subsubsection{登录接口}
URL:/api/lookupattribute

请求类型:GET

参数:

• path:文件路径

• cookie:无需用户手动输入

返回值:bool类型,表示是否成功

\subsubsection{压缩/解压缩接口}
URL:/api/zip

请求类型:POST

参数: 

• path:文件路径

• op:表示压缩还是解压缩

• cookie:无需用户手动输入

返回值:bool类型,表示是否成功.并返回显示的页面

\subsubsection{举报接口}
URL:/api/report

请求类型:POST

参数: 

• path:文件路径

• cookie:无需用户手动输入

返回值:bool类型,表示是否成功并返回显示的页面

\subsubsection{搜索接口}
URL:/api/search.do

请求类型:POST

参数: 

• path:文件路径

• cookie:无需用户手动输入

返回值:bool类型,表示是否成功.并返回显示的页面

\subsubsection{生成分享链接接口}
URL:/api/share.do

请求类型:POST

参数: 

• path:文件路径

• cookie:无需用户手动输入

返回值:链接的地址.并返回显示的页面

\subsubsection{导入链接接口}
URL:/api/getshare.do

请求类型:POST

参数: 

• link:要导入的链接

• path:文件夹

• cookie:无需用户手动输入

返回值:bool类型,表示是否成功.并返回显示的页面


\section{内部接口}

\subsection{MODULE.CLIENT.1: 通信模块}
\subsubsection{connection create\_connection()}
说明:建立连接

\subsection{MODULE.CLIENT.2: 登陆模块}

\subsubsection{user\_login(connection,loginfo)}
参数connection:浏览器和服务器间的连接
参数loginfo:用户登录信息
说明:用户登录账号
class loginfo:
String user\_name:用户名
String password:密码
\subsubsection{user\_register(connection, loginfo)}
参数connection:浏览器和服务器间的连接
参数loginfo:用户注册信息
说明:用户注册账号
\subsubsection{user\_find\_password(connection, user\_name, user\_email)}
参数connection:浏览器和服务器间的连接
参数user\_name:用户名
参数user\_email:用户注册时的电子邮箱

\subsection{MODULE.CLIENT.3: 标签页模块}

\subsubsection{add\_tab(connection)}
参数connection:浏览器和服务器间的连接
\subsubsection{close\_tab(connection, tabinfo)}
参数connection:浏览器和服务器间的连接
参数tabinfo:已经打开的标签页信息

\subsection{MODULE.CLIENT.4: 显示模块}
\subsubsection{show(connection)}
参数connection:浏览器和服务器间的连接

\subsection{MODULE.SERVER.5: 分享管理模块}
\subsubsection{create\_share(fileID, fetch\_password, due\_time}
参数fileID:要分享的文件的ID
参数fetch\_password:下载该资源需要的密码
参数due\_time:链接失效日期
\subsection{send\_share(shareInfo)}
class shareInfo:
    int owner\_id
    str link\_url
    str link\_passward
    time due\_time

\subsection{MODULE.SERVER.6: 收藏夹模块}
\subsubsection{create\_bookmark(connection)}
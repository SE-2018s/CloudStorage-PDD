\chapter{数据结构设计}

\section{逻辑结构设计}
使用伪代码来表示数据结构的设计
\subsection{文件数据结构}
\begin{lstlisting}[language=Python]
    class File:
        str file_id
        str file_name
        str file_mode
        str owner_name
        int num_bytes
        time last_updated
\end{lstlisting}

\subsection{用户信息数据结构}
\begin{lstlisting}[language=Python]
    class User:
        str user_id
        str name
        str passward
        str email
        list files_own

\end{lstlisting}
\subsection{链接数据结构}
\begin{lstlisting}[language=Python]
    class Link:
        str file_id
        str owner_id
        str link_url
        str link_passward
        time due_time

\end{lstlisting}

\section{物理结构设计}
各数据结构无特殊物理结构要求。

\section{数据结构与程序模块的关系}
[此处指的是不同的数据结构分配到哪些模块去实现。可按不同的端拆分此表]
\begin{table}[htbp]
\centering
\caption{数据结构与程序代码的关表} \label{tab:datastructure-module}
\begin{tabular}{|c|c|c|c|}
    \hline
    · & 模块1 & 模块2 & 模块3 \\
    \hline
    结构1 & · & Y & · \\
    \hline
    结构2 & · & Y & · \\
    \hline
    结构3 & · & Y & · \\
    \hline
    结构4 & Y & · & · \\
    \hline
    结构5 & · & · & Y \\
    \hline
\end{tabular}
\note{各项数据结构的实现与各个程序模块的分配关系}
\end{table}
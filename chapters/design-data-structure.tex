\chapter{数据结构设计}

\section{逻辑结构设计}
使用伪代码来表示数据结构的设计
\subsection{文件数据结构}
\begin{lstlisting}[language=Python]
    class File:
        str file_name
        str file_mode
        int owner_id
        int num_bytes
        time last_updated
        list right_list
\end{lstlisting}
说明:此类包含了云盘上文件的相关信息
\subsection{用户信息数据结构}
\begin{lstlisting}[language=Python]
    class User:
        str name
        str passward
        str email
        list files_own
\end{lstlisting}
说明:此类包含了用户的基本信息
\subsection{链接数据结构}
\begin{lstlisting}[language=Python]
    class Link:
        int owner_id
        str link_url
        str link_passward
        time due_time
\end{lstlisting}
说明:此类包含了分享云盘资源所需要的信息

\subsection{举报信息数据结构}
\begin{lstlisting}[language=Python]
    class Report:
        int file_id
        int report_type
        str report_detail
        bin report_fig
\end{lstlisting}
说明:此类包含了举报文件的信息
        
\section{物理结构设计}
各数据结构无特殊物理结构要求。

\section{数据结构与程序模块的关系}
\begin{table}[htbp]
\centering
\caption{数据结构与程序代码的关表} \label{tab:datastructure-module}
\begin{tabular}{|c|c|c|c|c|}
    \hline
    · & 用户结构 & 文件结构 & 链接结构 & 举报结构\\
    \hline
    客户端登录模块 & Y & · & · & ·\\
    \hline
    客户端标签页模块 & Y & · & · & ·\\
    \hline
    客户端生成模块 & Y & Y & · & ·\\
    \hline
    文件传输模块 & Y & · & Y & ·\\
    \hline
    服务器用户管理模块 & Y & · & · & ·\\
    \hline
    服务器权限加密模块 & Y & Y & · & ·\\
    \hline
    分享模块 & Y & Y & Y & ·\\
    \hline
    收藏夹模块 & Y & Y & · & ·\\
    \hline
    举报模块 & Y & Y & · & Y\\
    \hline
    回收站模块  & Y & Y & · & ·\\
    \hline
    解压压缩模块  & · & Y & · & ·\\
    \hline
    版本管理模块  & Y & Y & · & ·\\
    \hline
    共享文件夹模块 & Y & Y & · & ·\\
    \hline
\end{tabular}
\note{各项数据结构的实现与各个程序模块的分配关系}
\end{table}
\chapter{引言}
\section{编写目的}
在本项目的前一阶段,也就是需求分析阶段,已经将系统用户对本系统的需求做了详细的阐述,这些用户需求已经在上一阶段中对不同用户所提出的不同功能,实现的各种效果做了调研工作,并在需求规格说明书中得到详尽得叙述及阐明。

本阶段已在系统的需求分析的基础上,对xyz云盘系统进行设计。主要解决了实现该系统需求的程序模块设计问题。包括如何把该系统划分成若干个模块、决定各个模块之间的接口、模块之间传递的信息,以及数据结构、模块结构的设计等。在以下的概要设计报告中将对在本阶段中对系统所做的所有概要设计进行详细的说明,在设计过程中起到了提纲挈领的作用。

在下一阶段的详细设计中,程序设计员可参考此概要设计报告,在概要设计即时聊天工具所做的模块结构设计的基础上,对系统进行详细设计。在以后的软件测试以及软件维护阶段也可参考此说明书,以便于了解在概要设计过程中所完成的各模块设计结构,或在修改时找出在本阶段设计的不足或错误。


\section{项目背景}
随着互联网技术的飞速发展以及广泛应用,云计算这一技术也随之普及。云存储,是近几年在云计算的发展潮流之中诞生的,一项新兴的网络存储技术。云存储集成了网络技术和分布式文件系统等功能,是通过对不同的物理存储设备进行虚拟化映射,以形成逻辑层面统一的大存储空间的应用。

云盘系统,就是利用云存储技术,面向广大的有存储需求的客户,提供数据文件存储服务的第三方托管系统。

我们的xyz云盘系统,是基于分布式文件系统来设计和开发的云盘系统,是一个独立的项目。它的命名来自三位开发者名字的首字母(Xiaoniu,Yongzhou,Zewen),表明这将是由三位开发者开发的完全不同于其他云盘系统的新兴的云盘系统。

\section{术语}
\begin{table}[htbp]
\centering
\caption{术语表} \label{tab:terminology}
\begin{tabular}{|c|c|}
    \hline
    缩写、术语 & 解释 \\
    \hline
    CentOS & Community Enterprise Operating Syste , 社区事业版操作系统\\
    \hline
    CSS & Cascading Style Sheets , 层叠样式表 \\
    \hline
    HTML & HyperText Markup Language , 超文本标记语言 \\
    \hline
    HTTP & HyperText Transfer Protocol , 超文本传输协议 \\
    \hline
    HTTPS & Hypertext Transfer Protocol Secure , 超文本传输安全协议 \\ 
    \hline
    IOPS & Input/Output Operations Per Second  , 每秒读写操作的次数\\
    \hline
    IP & Internet Protocol , 网际协议\\
    \hline
    MD5 & Message-Digest Algorithm 5 , 讯息摘要演算法 5\\
    \hline
    TCP & Transmission Control Protocol , 传输控制协议\\
    \hline
\end{tabular}
% \note{这里是表的注释}
\end{table}
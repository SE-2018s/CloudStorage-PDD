\chapter{任务概述}
本系统的目标是实现一个xyz云盘系统,包括客户端、服务器端两个部分。

客户端面向xyz云盘用户,为用户提供p2p下载、上传、分享、加密、在线解压缩等服务。

\section{目标}
实现xyz云盘系统,实现需求规格说明书中所描述的p2p下载、上传、分享、备份、加密、审核、在线解压缩和预览等功能,并且保证系统的健壮性、高可用性和数据安全。

\section{开发与运行环境}

\subsection{开发环境的配置}
如表2.1开发环境的配置所示。
\begin{table}[htbp]
\centering
\caption{开发环境的配置} \label{tab:development-environment}
\begin{tabular}{|c|c|c|}
    \hline
    类别 & 标准配置 & 最低配置 \\
    \hline
    计算机硬件 & 基于x86结构的CPU & 基于x86结构的CPU \\ 
             & 内存>=8G & 内存>=4G \\
             & 硬盘>=250G & 硬盘>= 100G \\
             & 主频>=2.4GHz & 主频>=1.0GHz \\
             & 网络带宽>=100MBps & 网络带宽>=10MBps \\ 
    \hline
    计算机软件 & 服务器端:CentOS (version>=7.4) & 服务器端:CentOS (version>=7.1)\\
             & 客户端:Win10(version>=1709) & 客户端:Win10(version>=1703)\\
    \hline
    其他 & Java HotSpot VM 18.3 &  Java HotSpot VM 18.3\\
        & Tomcat (version>=9) &  Tomcat (version>=9)\\ 
        & Mysql(version>=5.7) &  Mysql(version>=5.6)\\
        & Google Chrome(version>=61)等浏览器 &  Google Chrome(version>=61)等浏览器\\
        & Ceph(version>=12.2) &  Ceph(version>=12.0)\\
    \hline
\end{tabular}
% \note{这里是表的注释}
\end{table}

\subsection{测试环境的配置}
如表2.2测试环境的配置所示。
\begin{table}[htbp]
\centering
\caption{测试环境的配置} \label{tab:test-environment}
\begin{tabular}{|c|c|c|}
    \hline
    类别 & 标准配置 & 最低配置 \\
    \hline
    计算机硬件 & 内存>=8G & 内存>=4G \\
    (服务器端)         & 硬盘>=250G & 硬盘>= 100G \\
             & 主频>=2.4GHz & 主频>=1.0GHz \\
             & 网络带宽>=1GBps & 网络带宽>=10MBps \\ 
    \hline
    计算机硬件 & 内存>=2G & 内存>=1G \\
    (客户端)         & 硬盘剩余空间>=1G & 硬盘剩余空间>= 100M \\
             & 主频>=2.4GHz & 主频>=1.0GHz \\
             & 网络带宽>=100MBps & 网络带宽>=10MBps \\ 
    \hline
    计算机软件 & 服务器端:CentOS (version>=7.4) & 服务器端:CentOS (version>=7.1)\\
             & 客户端:Win10(version>=1709) & 客户端:Win10(version>=1703)\\
    \hline
    其他 & Java HotSpot VM 18.3 &  Java HotSpot VM 18.3\\
        & Tomcat (version>=9) &  Tomcat (version>=9)\\ 
        & Mysql(version>=5.7) &  Mysql(version>=5.6)\\
        & Google Chrome(version>=61)等浏览器 &  Google Chrome(version>=61)等浏览器\\
        & Ceph(version>=12.2) &  Ceph(version>=12.0)\\
    \hline 

\end{tabular}
% \note{这里是表的注释}
\end{table}

\subsection{运行环境的配置}
如表2.3运行环境的配置所示。
\begin{table}[htbp]
\centering
\caption{运行环境的配置} \label{tab:operation-environment}
\begin{tabular}{|c|c|c|}
    \hline
    类别 & 标准配置 & 最低配置 \\
    \hline
    计算机硬件 & 内存>=8G & 内存>=4G \\
    (服务器端)         & 硬盘>=250G & 硬盘>= 100G \\
             & 主频>=2.4GHz & 主频>=1.0GHz \\
             & 网络带宽>=1GBps & 网络带宽>=10MBps \\ 
    \hline
    计算机硬件 & 内存>=2G & 内存>=1G \\
    (客户端)         & 硬盘剩余空间>=1G & 硬盘剩余空间>= 100M \\
             & 主频>=2.4GHz & 主频>=1.0GHz \\
             & 网络带宽>=100MBps & 网络带宽>=500KBps \\ 
    \hline
    计算机软件 & 服务器端:CentOS (version>=7.4) & 服务器端:CentOS (version>=7.1)\\
             & 客户端:Win10(version>=1709) & 客户端:Win10(version>=1703)\\
    \hline
    其他 & Java HotSpot VM 18.3 &  Java HotSpot VM 18.3\\
        & Tomcat (version>=9) &  Tomcat (version>=9)\\ 
        & Mysql(version>=5.7) &  Mysql(version>=5.6)\\
        & Google Chrome(version>=61)等浏览器&  Google Chrome(version>=61)等浏览器\\
        & Ceph(version>=12.2) &  Ceph(version>=12.0)\\
    \hline
\end{tabular}
% \note{这里是表的注释}
\end{table}

\section{需求概述}
功能需求包括: 

用户的注册、登陆、退出、忘记密码。

文件的上传、下载、重命名、移动、加密、分享、搜索、在线解压缩、审核、共享文件夹等。

\section{条件与限制}
为了完成这个项目,xyz云盘系统的的开发应该在以下条件下展开:

• 开发者掌握足够的开发xyz云盘系统的能力,比如前后端代码编写的能力、UI设计的能力、与客户进行有效沟通的能力等。

• 开发者掌握足够的需要开发xyz云盘系统的软硬件环境配置,尤其是开发过程中的软硬件资源、以及运行时的足够的服务器资源。
 
• 开发者有足够的精力与时间进行xyz云盘系统的开发。

• 开发者能够负责后续的项目更新、bug修复等事宜。

同时,xyz云盘系统的开发具有如下的限制因素:

• 硬件资源:开发者没有足够的经费以维持足够的服务器硬件开销,尤其是庞大的硬盘开销以及网络带宽开销。

• 开发经验:开发者没有足够的相关应用的开发经验。

• 人力资源:没有足够的人员数量以及开发的时间精力。

• 用户隐私:xyz云盘系统为了在中国能够合法的运营下去,必须遵守中国的相关法律规定,包括存储必要的用户数据以允许相关部门的合法审查,而这必然会限制用户隐私的绝对保护。

• 安全性依赖:xyz云盘系统依赖于Windows、Java、CentOS、Mysql、Ceph、Tomcat、Chrome等多个第三方开源产品,所以其安全性受到这些第三方产品的限制。

• 性能依赖:xyz云盘系统由于是网盘软件,所以其下载、上传等功能会极大地受到用户自身硬件资源等的限制,所以性能与用户硬件的相关性极大。

• 浏览器依赖:xyz云盘系统的客户端部署在用户的浏览器中,所以根据浏览器对Http、Https等协议的实现不同,其兼容性可能也会有些许偏差。
